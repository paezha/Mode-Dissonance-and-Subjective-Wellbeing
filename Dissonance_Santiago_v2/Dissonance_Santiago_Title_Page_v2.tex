\documentclass[]{elsarticle} %review=doublespace preprint=single 5p=2 column
%%% Begin My package additions %%%%%%%%%%%%%%%%%%%
\usepackage[hyphens]{url}

  \journal{Journal of Transportation \& Health} % Sets Journal name


\usepackage{lineno} % add
  \linenumbers % turns line numbering on
\providecommand{\tightlist}{%
  \setlength{\itemsep}{0pt}\setlength{\parskip}{0pt}}

\usepackage{graphicx}
\usepackage{booktabs} % book-quality tables
%%%%%%%%%%%%%%%% end my additions to header

\usepackage[T1]{fontenc}
\usepackage{lmodern}
\usepackage{amssymb,amsmath}
\usepackage{ifxetex,ifluatex}
\usepackage{fixltx2e} % provides \textsubscript
% use upquote if available, for straight quotes in verbatim environments
\IfFileExists{upquote.sty}{\usepackage{upquote}}{}
\ifnum 0\ifxetex 1\fi\ifluatex 1\fi=0 % if pdftex
  \usepackage[utf8]{inputenc}
\else % if luatex or xelatex
  \usepackage{fontspec}
  \ifxetex
    \usepackage{xltxtra,xunicode}
  \fi
  \defaultfontfeatures{Mapping=tex-text,Scale=MatchLowercase}
  \newcommand{\euro}{€}
\fi
% use microtype if available
\IfFileExists{microtype.sty}{\usepackage{microtype}}{}
\bibliographystyle{elsarticle-harv}
\ifxetex
  \usepackage[setpagesize=false, % page size defined by xetex
              unicode=false, % unicode breaks when used with xetex
              xetex]{hyperref}
\else
  \usepackage[unicode=true]{hyperref}
\fi
\hypersetup{breaklinks=true,
            bookmarks=true,
            pdfauthor={},
            pdftitle={Do Drivers Dream of Walking? An Investigation of Travel Mode Dissonance from the Perspective of Subjective Wellbeing},
            colorlinks=false,
            urlcolor=blue,
            linkcolor=magenta,
            pdfborder={0 0 0}}
\urlstyle{same}  % don't use monospace font for urls

\setcounter{secnumdepth}{5}
% Pandoc toggle for numbering sections (defaults to be off)


% Pandoc header



\begin{document}
\begin{frontmatter}

  \title{Do Drivers Dream of Walking? An Investigation of Travel Mode Dissonance
from the Perspective of Subjective Wellbeing}
    \author[PUC]{Beatriz Mella Lira\corref{Corresponding Author}}
   \ead{beatriz.lira.14@ucl.ac.uk} 
    \author[McMaster University]{Antonio Paez}
   \ead{paezha@mcmaster.ca} 
      \address[PUC]{BRT+ Centre of Excellence, Departamento Ingeniería de Transporte y
Logística, Escuela de Ingeniería, Pontificia Universidad Católica de
Chile}
    \address[McMaster University]{School of Earth, Environment, and Society, McMaster University, 1280
Main St W, Hamilton, Ontario, L8S 1S4 Canada}
    
  \begin{abstract}
  \emph{Introduction}\\
  Subjective wellbeing is a topic that has attracted considerable
  attention in the transportation literature in recent years. As a result,
  there is a burgeoning literature that investigates the impacts of travel
  on subjective wellbeing, and how wellbeing, in turn, can influence
  behavior. An important aspect of subjective wellbeing are the affective
  reactions of people to their experiences.\\
  \emph{Objective}\\
  The objective of this paper is to analyze the affective reactions of
  travelers with respect to various modes of transportation. In
  particular, we are interested in the potential for dissonance between
  primary mode of travel and the mode(s) of travel identified as evoking
  various affective reactions.\\
  \emph{Materials and Methods}\\
  The study is based on data collected from a sample of travelers in the
  city of Santiago, in Chile. Participants in the study were asked about
  their usual mode of travel, and then were asked to name their ideal
  mode(s) of transportation from the perspective of various affective
  reactions. The reactions we investigate are associated with the values
  of freedom, enjoyment, happiness, poverty, luxury, and status. Analysis
  is based on tests of independence and visualization techniques.\\
  \emph{Results}\\
  The results indicate that users of public transportation experience the
  most dissonance in terms of affective reactions, and active travelers
  the least. For those travelers who experience dissonance, active travel
  is the mode most commonly associated with freedom, enjoyment, and
  happiness, whereas public transportation is most commonly associated
  with poverty. The automobile, in contrast, is the mode most commonly
  associated with luxury and status.
  \end{abstract}
  
 \end{frontmatter}




\end{document}


