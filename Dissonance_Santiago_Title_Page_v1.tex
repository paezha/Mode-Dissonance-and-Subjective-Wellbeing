\documentclass[]{elsarticle} %review=doublespace preprint=single 5p=2 column
%%% Begin My package additions %%%%%%%%%%%%%%%%%%%
\usepackage[hyphens]{url}

  \journal{An Elsevier journal} % Sets Journal name


\usepackage{lineno} % add
\providecommand{\tightlist}{%
  \setlength{\itemsep}{0pt}\setlength{\parskip}{0pt}}

\usepackage{graphicx}
\usepackage{booktabs} % book-quality tables
%%%%%%%%%%%%%%%% end my additions to header

\usepackage[T1]{fontenc}
\usepackage{lmodern}
\usepackage{amssymb,amsmath}
\usepackage{ifxetex,ifluatex}
\usepackage{fixltx2e} % provides \textsubscript
% use upquote if available, for straight quotes in verbatim environments
\IfFileExists{upquote.sty}{\usepackage{upquote}}{}
\ifnum 0\ifxetex 1\fi\ifluatex 1\fi=0 % if pdftex
  \usepackage[utf8]{inputenc}
\else % if luatex or xelatex
  \usepackage{fontspec}
  \ifxetex
    \usepackage{xltxtra,xunicode}
  \fi
  \defaultfontfeatures{Mapping=tex-text,Scale=MatchLowercase}
  \newcommand{\euro}{€}
\fi
% use microtype if available
\IfFileExists{microtype.sty}{\usepackage{microtype}}{}
\bibliographystyle{elsarticle-harv}
\ifxetex
  \usepackage[setpagesize=false, % page size defined by xetex
              unicode=false, % unicode breaks when used with xetex
              xetex]{hyperref}
\else
  \usepackage[unicode=true]{hyperref}
\fi
\hypersetup{breaklinks=true,
            bookmarks=true,
            pdfauthor={},
            pdftitle={Do Drivers Dream of Walking? An Investigation of Travel Mode Dissonance from the Perspective of Subjective Wellbeing},
            colorlinks=false,
            urlcolor=blue,
            linkcolor=magenta,
            pdfborder={0 0 0}}
\urlstyle{same}  % don't use monospace font for urls

\setcounter{secnumdepth}{0}
% Pandoc toggle for numbering sections (defaults to be off)
\setcounter{secnumdepth}{0}
% Pandoc header



\begin{document}
\begin{frontmatter}

  \title{Do Drivers Dream of Walking? An Investigation of Travel Mode Dissonance
from the Perspective of Subjective Wellbeing}
    \author[University College London]{Beatriz Mella Lira\corref{c1}}
   \ead{beatriz.lira.14@ucl.ac.uk} 
   \cortext[c1]{Corresponding Author}
    \author[McMaster University]{Antonio Paez}
   \ead{paezha@mcmaster.ca} 
  
      \address[University College London]{Bartlett School of Planning, Sixth Floor, Central House, 14 Upper Woburn
Place, London, WC1H0NN UK}
    \address[McMaster University]{School of Geography and Earth Sciences, 1280 Main St W, Hamilton,
Ontario, L8S 1S4 Canada}
  
  \begin{abstract}
  Transportation in most of the world has been dominated for decades by a
  fascination with the automobile. Nonetheless, there is an increasing
  recognition that to achieve a variety of economic, environmental, and
  public health policy goals it is important to attract and retain active
  travelers and users of public transportation. A challenge with the way
  transportation policy is developed, however, is that it tends to focus
  on utilitarian considerations, which may miss other potential aspects of
  transportation that users value. In particular, subjective wellbeing
  (SWB) has been proposed as a way to enhance our understanding of the
  preferences and choices of travellers, as well as a way to evaluate the
  benefits of transportation beyond utilitarian considerations. The
  objective of this paper is to analyze the modes that people commonly
  use, and to what extent they are aligned (or not) with a variety of
  affective values associated with SWB. In other words, we are interested
  in the potential for dissonance with respect to the primary mode of
  travel, from the perspective of affective values. The study is based on
  data collected from a sample of travellers in the city of Santiago, in
  Chile. Participants in the study were asked about their usual mode of
  travel, and then were asked to name the mode or modes that they
  associate with the affective values of freedom, enjoyment, happiness,
  poverty, luxury and status. The results indicate that users of public
  transportation experience the most dissonance in terms of affective
  values, and active travellers the least. For those travellers who
  experience dissonance, active travel is the mode most commonly
  associated with freedom, enjoyment, and happiness, public transportation
  is most commonly associated with poverty, and the automobil is most
  commonly associated with luxury and status.
  \end{abstract}
  
 \end{frontmatter}




\end{document}


