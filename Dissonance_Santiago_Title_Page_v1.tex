\documentclass[]{elsarticle} %review=doublespace preprint=single 5p=2 column
%%% Begin My package additions %%%%%%%%%%%%%%%%%%%
\usepackage[hyphens]{url}

  \journal{Journal of Transportation \& Health} % Sets Journal name


\usepackage{lineno} % add
  \linenumbers % turns line numbering on
\providecommand{\tightlist}{%
  \setlength{\itemsep}{0pt}\setlength{\parskip}{0pt}}

\usepackage{graphicx}
\usepackage{booktabs} % book-quality tables
%%%%%%%%%%%%%%%% end my additions to header

\usepackage[T1]{fontenc}
\usepackage{lmodern}
\usepackage{amssymb,amsmath}
\usepackage{ifxetex,ifluatex}
\usepackage{fixltx2e} % provides \textsubscript
% use upquote if available, for straight quotes in verbatim environments
\IfFileExists{upquote.sty}{\usepackage{upquote}}{}
\ifnum 0\ifxetex 1\fi\ifluatex 1\fi=0 % if pdftex
  \usepackage[utf8]{inputenc}
\else % if luatex or xelatex
  \usepackage{fontspec}
  \ifxetex
    \usepackage{xltxtra,xunicode}
  \fi
  \defaultfontfeatures{Mapping=tex-text,Scale=MatchLowercase}
  \newcommand{\euro}{€}
\fi
% use microtype if available
\IfFileExists{microtype.sty}{\usepackage{microtype}}{}
\bibliographystyle{elsarticle-harv}
\ifxetex
  \usepackage[setpagesize=false, % page size defined by xetex
              unicode=false, % unicode breaks when used with xetex
              xetex]{hyperref}
\else
  \usepackage[unicode=true]{hyperref}
\fi
\hypersetup{breaklinks=true,
            bookmarks=true,
            pdfauthor={},
            pdftitle={Do Drivers Dream of Walking? An Investigation of Travel Mode Dissonance from the Perspective of Subjective Wellbeing},
            colorlinks=false,
            urlcolor=blue,
            linkcolor=magenta,
            pdfborder={0 0 0}}
\urlstyle{same}  % don't use monospace font for urls

\setcounter{secnumdepth}{5}
% Pandoc toggle for numbering sections (defaults to be off)
% Pandoc header



\begin{document}
\begin{frontmatter}

  \title{Do Drivers Dream of Walking? An Investigation of Travel Mode Dissonance
from the Perspective of Subjective Wellbeing}
    \author[University College London]{Beatriz Mella Lira\corref{c1}}
   \ead{beatriz.lira.14@ucl.ac.uk} 
   \cortext[c1]{Corresponding Author}
    \author[McMaster University]{Antonio Paez}
   \ead{paezha@mcmaster.ca} 
  
      \address[University College London]{Bartlett School of Planning, Sixth Floor, Central House, 14 Upper Woburn
Place, London, WC1H0NN UK}
    \address[McMaster University]{School of Geography and Earth Sciences, 1280 Main St W, Hamilton,
Ontario, L8S 1S4 Canada}
  
  \begin{abstract}
  \textbf{Introduction}\\
  Subjective wellbeing is a topic that has attracted considerable
  attention in recent years due to the way it correlates with health. From
  a transportation perspective, there is a burgeoning literature on the
  way travel can impact subjective wellbeing, and how this, in turn, can
  influence behavior.\\
  \textbf{Objective}\\
  The objective of this paper is to analyze a number of affective values
  associated with subjective wellbeing and the modes of transportation
  that people commonly use. In particular, we are interested in the
  potential for dissonance with respect to the primary mode of travel.\\
  \textbf{Materials and Methods}\\
  The study is based on data collected from a sample of travellers in the
  city of Santiago, in Chile. Participants in the study were asked about
  their usual mode of travel, and then were asked to name the mode or
  modes that they associate with the affective values of freedom,
  enjoyment, happiness, poverty, luxury and status. Analysis is based on
  tests of independence and visualization via mosaic plots.\\
  \textbf{Results}\\
  The results indicate that users of public transportation experience the
  most dissonance in terms of affective values, and active travellers the
  least. For those travellers who experience dissonance, active travel is
  the mode most commonly associated with freedom, enjoyment, and
  happiness, public transportation is most commonly associated with
  poverty, and the automobil is most commonly associated with luxury and
  status.
  \end{abstract}
  
 \end{frontmatter}




\end{document}


